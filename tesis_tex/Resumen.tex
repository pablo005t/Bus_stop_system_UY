\chapter*{Resumen}

El presente trabajo desarrolla un prototipo operativo para la identificación y gestión de paradas oficiales de transporte público en rutas nacionales del Uruguay, integrando técnicas de Visión por Computadora (\textit{Computer Vision}, CV) y análisis geoespacial. La motivación surge a partir de una convocatoria de la Agencia Nacional de Investigación e Innovación (ANII) y el Ministerio de Transporte y Obras Públicas (MTOP), orientada a relevar y diagnosticar el estado actual de las paradas suburbanas. Si bien no se accedió a los datos de \textit{street view} previstos en el llamado, se decidió continuar con la investigación utilizando imágenes satelitales, lo que permitió adaptar la metodología al entorno disponible y explorar una alternativa viable para la identificación automática de paradas. La propuesta integra modelos de detección aplicados sobre imágenes satelitales, datos geoespaciales de infraestructura vial y variables demográficas del Instituto Nacional de Estadística (INE) y de la Administración Nacional de Educación Pública (ANEP), como la población por localidad y centros educativos, con el fin de estimar la demanda potencial de transporte. Como resultado, se desarrolla un \textit{pipeline} reproducible que combina detección, verificación y un sistema de priorización basado en criterios geográficos y demográficos. Este enfoque busca servir como propuesta de mejora para el sistema actual de registro y planificación de paradas en rutas nacionales, aportando una herramienta de apoyo a la gestión pública y demostrando el potencial de la visión por computadora aplicada al ordenamiento territorial.
